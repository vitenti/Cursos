\newif\ifuseseminar
\useseminartrue
\input{../../latex_common/header.tex.frag}

\title{Mecânica Quântica II -- Spin e Momento Angular}

\begin{document}
\begin{enumerate}
	\item Dadas as matrizes de Pauli abaixo, responda às questões.
	      \begin{align*}
		      \sigma_x & = \begin{pmatrix}
			                   0 & 1 \\
			                   1 & 0 \\
		                   \end{pmatrix}, &
		      \sigma_y & = \begin{pmatrix}
			                   0 & -i \\
			                   i & 0  \\
		                   \end{pmatrix}, &
		      \sigma_z = \begin{pmatrix}
			                 1 & 0  \\
			                 0 & -1 \\
		                 \end{pmatrix}.
	      \end{align*}
	      \begin{enumerate}
		      \item Explique qual é a relação entre essas matrizes e os elementos de
		            matriz dos operadores de Spin $\hat{S}_x,\,\hat{S}_y,\,\hat{S}_z$.
		      \item Descreva como os elementos de matriz estão relacionados à base
		            $\ket{j,m}$, onde $j$ é o rótulo do momento angular total
		            $\hat{\vec{S}}^2$ e $m$ é o rótulo da projeção do momento angular ao longo
		            do eixo $z$, ou seja, $\hat{S}_z$.
		      \item Explique como é possível utilizar as matrizes de Pauli apresentadas
		            para calcular explicitamente os elementos do grupo de rotação. Utilize a
		            fórmula explícita abaixo como referência:
		            $$ U[R] = e^{-i\vec{\theta}\cdot\vec{S}/\hbar} $$ Descreva o
		            procedimento passo a passo e discuta como as propriedades das
		            matrizes de Pauli estão relacionadas com o resultado.
		      \item Utilizando a representação dos elementos do grupo de rotação,
		            encontre as duas rotações necessárias para transformar o vetor unitário na
		            direção $z$, $\vec{e}_3$, em um vetor unitário $\vec{n} =
			            (\sin\theta\cos\phi,\, \sin\theta\sin\phi,\, \cos\theta)$. Em seguida,
		            usando essas rotações, encontre o Spinor que representa um estado de Spin
		            positivo na direção $\vec{n}$.
	      \end{enumerate}
	\item Considere o termo de interação dado por $H_\mathrm{int} = -\gamma
		      \vec{S}\cdot\vec{B}$, onde $\vec{S}$ é o vetor de spin e $\vec{B}$ é um campo
	      magnético constante na direção $\vec{e}_3$. Responda às seguintes questões:
	      \begin{enumerate}
		      \item Dado um Spinor inicial $\psi_0$ que é autovetor de
		            $\vec{n}\cdot\vec{S}$, descreva como ocorre a evolução desse estado
		            considerando apenas o termo de interação como a Hamiltoniana. Calcule
		            explicitamente $\psi(t)$.
		      \item Calcule os autovetores de $\sigma_x$ e $\sigma_y$, usando esses
		            estados determine a probabilidade de encontrar o spin positivo nas
		            direções $x$ e $y$ em função do tempo. Interprete o resultado fisicamente.
	      \end{enumerate}
	\item Adição de momento angular: Considere duas representações irredutíveis
	      rotuladas por $j_1$ e $j_2$, ou seja, $\mathbb{V}_{j_1}$ e $\mathbb{V}_{j_2}$. O
	      espaço formado pelo produto tensorial dessas duas representações
	      ($\mathbb{V}_{j_1}\otimes\mathbb{V}_{j_2}$) não é irredutível, portanto, pode ser
	      decomposto como uma soma direta de representações irredutíveis. Essas representações
	      estão relacionadas ao operador de momento angular total $\hat{\vec{J}} =
		      \hat{\vec{J}}_1 + \hat{\vec{J}}_2$. Com base nessas informações, responda às
	      seguintes perguntas:
	      \begin{enumerate}
		      \item Quais são os possíveis valores para o momento angular total que
		            podem aparecer na soma direta? (Leve em consideração os possíveis valores
		            de $\hat{J}_z$).
		      \item Quantas vezes cada um desses valores do momento angular total
		            aparece na soma direta? (Conte quantas vezes o mesmo autovalor de
		            $\hat{J}_z$ pode ser encontrado).
		      \item Usando os resultados anteriores escreva explicitamente a
		            decomposição de $\mathbb{V}_{2}\otimes\mathbb{V}_{3/2}$ em uma soma
		            direta.
		      \item Considere uma partícula composta em um estado de Spin total $s=0$
		            que decai em duas partículas de Spin $s_1=s_2=1/2$. Qual é o estado final
		            do sistema na representação $\mathbb{V}_{1/2}\otimes\mathbb{V}_{1/2}$?
		            Desconsidere qualquer momento angular orbital.
	      \end{enumerate}
\end{enumerate}

\end{document}

\newif\ifuseseminar
\useseminartrue
\input{../../latex_common/header.tex.frag}

\title{Mecânica Quântica II -- Partículas Idênticas e Transformações}

\begin{document}
\begin{enumerate}
	\item Considere um sistema de duas partículas idênticas e responda às questões
	      abaixo:
	      \begin{enumerate}
		      \item Considere que o estado de uma partícula é descrito por $\ket{a} \in
			            \mathbb{V}$. Utilize o produto tensorial para definir os dois possíveis
		            espaços para os estados de duas partículas.
		      \item Definimos um estado separável de duas partículas como sendo da forma
		            $\ket{\psi} = \ket{a}\otimes\ket{b}$. Os estados que não podem ser
		            escritos dessa forma são chamados emaranhados. Os estados representando
		            duas partículas idênticas podem ser separáveis? Justifique sua resposta.
		      \item Com base nas respostas dos itens anteriores, explique como podemos
		            realizar experimentos em laboratório medindo o estado de uma partícula
		            $\ket{a}$ sem considerar todas as outras partículas idênticas presentes no
		            sistema.
		      \item Explique como podemos testar empiricamente se o estado de duas
		            partículas é simétrico ou anti-simétrico. Descreva como a simetria e a
		            anti-simetria estão relacionadas ao conceito de troca de partículas
		            idênticas e como as técnicas experimentais podem ser usadas para medir os
		            estados de duas partículas e verificar sua simetria.
	      \end{enumerate}
	\item Considere um grupo aditivo de transformações a um parâmetro, que é um mapa
	      suave $\mathbb{R} \to \mathrm{Op}(\mathbb{V})$ dado por $T(\epsilon)$ que satisfaz
	      as seguintes propriedades:
	      \begin{itemize}
		      \item $T(\epsilon_2)T(\epsilon_1) = T(\epsilon_1+\epsilon_2)$, onde
		            $\epsilon_1,\epsilon_2 \in \mathbb{R}$;
		      \item $T(0)$ é a identidade, ou seja, $T(0)=\mathbbm{1}$;
		      \item $T^\dagger(\epsilon)T(\epsilon)=\mathbbm{1}$, onde $T^\dagger$ é o adjunto de $T$.
	      \end{itemize}
	      Responda as seguintes questões:
	      \begin{enumerate}
		      \item Mostre que $T^\dagger(\epsilon) = T(-\epsilon)$.
		      \item Explique o conceito de gerador e mostre que o operador gerador é
		            anti-Hermitiano. Como podemos definir um operador que seja Hermitiano?
		      \item Considere uma função dos operadores básicos $\widehat{X}$ e
		            $\widehat{P}$, $\Omega(\widehat{X}, \widehat{P})$. Explique como podemos
		            reescrever a operação $T^\dagger(\epsilon)\Omega(\widehat{X},
			            \widehat{P})T(\epsilon)$.
	      \end{enumerate}
	\item Considere o operador de inversão espacial $\Pi$, que inverte as coordenadas
	      espaciais de um sistema. Prove que $\Pi$ é um operador idempotente, ou seja,
	      $\Pi^2 = \mathbbm{1}$. Em termos de funções de onda, quais são as auto-funções
	      do operador $\Pi$ e seus autovalores correspondentes? Além disso, suponha que
	      a Hamiltoniana do sistema é invariante sob inversão espacial, ou seja, $\Pi
		      \widehat{H} \Pi = \widehat{H}$. Nesse caso, quais características da função de
	      onda são conservadas? Explique.
	\item Considere a álgebra dos geradores de rotação $\widehat{\vec{L}} \times
		      \widehat{\vec{L}} = i\hbar\widehat{\vec{L}}$, e seja $\widehat{L}^2$ o operador
	      momento angular total e $\widehat{L}_z$ o operador componente $z$ do momento
	      angular. Suponha que existem autoestados comuns $\ket{\alpha\beta}$ de
	      $\widehat{L}^2$ e $\widehat{L}_z$, ou seja, $\widehat{L}^2\ket{\alpha\beta} =
		      \alpha\ket{\alpha\beta}$ e $\widehat{L}_z\ket{\alpha\beta} =
		      \beta\ket{\alpha\beta}$. Responda às seguintes perguntas:
	      \begin{enumerate}
		      \item Explique por que podemos definir um autoestado simultâneo de
		            $\widehat{L}^2$ e $\widehat{L}_z$.
		      \item Defina os operadores de escada $\widehat{L}_+$ e $\widehat{L}_-$ e
		            calcule seus comutadores com $\widehat{L}_i$ e $\widehat{L}^2$.
		      \item Mostre como os operadores de escada atuam nos estados $\ket{\alpha\beta}$.
		      \item Se $\widehat{L}_i$ são auto-adjuntos, mostre que deve existir uma
		            relação entre $\alpha$ e $\beta$. Qual é o significado físico dessa
		            relação?
	      \end{enumerate}
\end{enumerate}

\end{document}

\newif\ifuseseminar
\useseminartrue
\input{../../latex_common/header.tex.frag}

\title{Mecânica Quântica II -- Álgebra Linear}

\begin{document}
\begin{enumerate}
	\item Considere o conjunto $\mathbb{R}^n$, como podemos torná-lo um espaço vetorial?
	      Faça os seguintes itens:
	      \begin{enumerate}
		      \item Uma operação de soma de vetores.
		      \item Um operação de multiplicação por escalar.
		      \item Mostre que as propriedades de um espaço vetorial são satisfeitas.
	      \end{enumerate}
	\item Podemos transformar o conjunto das funções continuas entre $[0,L]$ (i.e.,
	      $C[0,L]$) em um espaço vetorial? Faça os seguintes itens:
	      \begin{enumerate}
		      \item Uma operação de soma de vetores.
		      \item Um operação de multiplicação por escalar.
		      \item Mostre que as propriedades de um espaço vetorial são satisfeitas.
		      \item Se restringirmos o conjunto original para funções em $C[0,L]$ que
		            satisfazem $f(0) = 1$ e $f(L) = 0$. As mesmas operações acima definem um
		            espaço vetorial? Por que?
	      \end{enumerate}
	\item Dado um espaço vetorial $\mathbb{V}^n$ com produto interno, mostre que o
	      conjunto dos vetores ortogonais a $|v\rangle$ forma um subespaço vetorial.
	\item Dados dois operadores lineares em no espaço vetorial com produto interno
	      $\mathbb{V}$ , i.e., $A,\,B\in \mathrm{Op}(\mathbb{V})$.
	      \begin{enumerate}
		      \item Mostre que se $A$ for auto-adjunto seus autovetores tem autovalores
		            reais.
		      \item Mostre que se $A$ for auto-adjunto seus autovetores com autovalores
		            diferentes são ortogonais.
		      \item Suponha que $B$ é auto-adjunto e $[A,B] = 0$. Mostre que se
		            $|a_i\rangle$ é autovetor de $A$, então $B|a_i\rangle$ também será.
	      \end{enumerate}

\end{enumerate}

\end{document}

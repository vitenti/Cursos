\newif\ifuseseminar
\useseminartrue
\input{../../latex_common/header.tex.frag}

\title{Mecânica Quântica II -- Revisão e Partículas Idênticas}

\begin{document}
\begin{enumerate}
	\item Oscilador Harmônico
	      \begin{enumerate}
		      \item Dado os operadores de aniquilação e criação definidos em termos dos
		            operadores de posição e momento:
		            \begin{align}
			            \hat{a}^\dagger = \sqrt{\frac{m\omega}{2\hbar}}\left(\hat{x} -
			            \frac{i}{m\omega}\hat{p}\right), \qquad \hat{a} =
			            \sqrt{\frac{m\omega}{2\hbar}}\left(\hat{x} + \frac{i}{m\omega}\hat{p}\right).
		            \end{align}
		            Calcule o comutador desses operadores e mostre que a Hamiltoniana
		            pode ser escrita da forma:
		            \begin{equation}
			            \hat{H} = \frac{\hat{p}^2}{2m} + \frac{m\omega^2\hat{x}^2}{2}=
			            \hbar\omega \left( \hat{a}^\dagger \hat{a} + \frac{1}{2} \right).
		            \end{equation}
		      \item A partir da equação de auto-estado da Hamiltoniana, $\hat{H}\ket{E}
			            = E\ket{E}$ use o comutador de $\hat{H}$ com os operadores de criação e
		            aniquilação mostre quais são os possíveis valores de $E$. Note que um
		            passo importante é impor que a Hamiltoniana tenha autovalores positivos
		            definidos, como isso pode ser justificado usando a Eq.~(2)?
	      \end{enumerate}
	\item Átomo de Hidrogênio
	      \begin{enumerate}
		      \item A partir do Hamiltoniano do átomo de hidrogênio $\hat{H} =
			            -\frac{\hbar^2}{2m} \nabla^2 - \frac{e^2}{4\pi\epsilon_0 r}$, descreva a
		            dedução dos autoestados da energia. Uma descrição qualitativa dos passos é
		            suficiente.
		      \item Quais são as propriedades da geometria do sistema, em termos das
		            coordenadas $(r,\theta,\phi)$, que limitam os autovalores de $\hat{H}$ e
		            como essa limitação se dá?
		      \item Quando estudamos a parte radial da solução, quais considerações
		            físicas sobre a função de onda são necessárias para encontrar os estados
		            ligados?
	      \end{enumerate}
	\item Considere um sistema de duas partículas {\bf idênticas} e responda às questões
	      abaixo:
	      \begin{enumerate}
		      \item Considere que o estado de uma partícula é descrito por $\ket{a} \in
			            \mathbb{V}$. Utilize o produto tensorial para definir os dois possíveis
		            espaços para os estados de duas partículas.
		      \item Definimos um estado separável de duas partículas como sendo da forma
		            $\ket{\psi} = \ket{a}\otimes\ket{b}$. Os estados que não podem ser
		            escritos dessa forma são chamados emaranhados. Os estados representando
		            duas partículas idênticas podem ser separáveis? Justifique sua resposta.
		      \item Com base nas respostas anteriores, explique como podemos realizar
		            experimentos em laboratório para medir o estado de uma partícula
		            $\ket{a}$ sem considerar todas as outras partículas idênticas presentes no
		            sistema. Para tanto, considere dois estados $\ket{T}$ e $\ket{L}$
		            representando pacotes gaussianos na Terra e na Lua, respectivamente, onde
		            $$\braket{x|T} =
			            \frac{1}{(2\pi)^{1/4}}\exp\left(-\frac{(x-\mu_T)^2}{4}\right),
			            \qquad \braket{x|L} =
			            \frac{1}{(2\pi)^{1/4}}\exp\left(-\frac{(x-\mu_L)^2}{4}\right),$$ com
		            $\mu_L\gg\mu_T$. Demonstre que tanto o estado simétrico quanto o
		            anti-simétrico de $\ket{L}$ e $\ket{T}$ são aproximadamente
		            separáveis quando calculamos a função de onda do conjunto em $(x_L,
			            x_T)$ onde $|x_T-\mu_L|\gg1$ e $|x_L-\mu_T|\gg1$.
		      \item Explique como podemos testar empiricamente se o estado de duas
		            partículas é simétrico ou anti-simétrico. Descreva como a simetria e a
		            anti-simetria estão relacionadas ao conceito de troca de partículas
		            idênticas e como as técnicas experimentais podem ser usadas para medir os
		            estados de duas partículas e verificar sua simetria.
	      \end{enumerate}

\end{enumerate}

\end{document}

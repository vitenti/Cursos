\newif\ifuseseminar
\useseminartrue
\input{../../latex_common/header.tex.frag}

\title{Mecânica Quântica I -- Postulados e Pacote de Ondas}

\begin{document}
\begin{enumerate}
      \item Postulados da Mecânica:
            \begin{enumerate}
                  \item Discorra sobre o estado de um sistema quântico. Compare com o
                        estado de um sistema clássico. Quais são as diferenças e
                        semelhanças?
                  \item Explique como são representados os observáveis em Mecânica
                        Quântica. O que é um operador hermitiano? Qual a relação entre
                        observáveis clássicos e quânticos?
                  \item Como a medida de um observável é representada em Mecânica
                        Quântica? O que é o valor esperado de um observável? Como ele é
                        calculado?
                  \item Compare e explique as diferenças entre a evolução temporal de um
                        sistema quântico e um sistema clássico. O que é o operador de
                        evolução temporal?
            \end{enumerate}
      \item Pacote de onda Gaussiano. Seja a condição inicial do sistema descrita por:
            \begin{equation}
                  \psi_0(x, t) = \left(\frac{1}{2\pi \sigma^2}\right)^{1/4}
                  \exp\left[-\frac{(x - x_0)^2}{4\sigma^2} + \frac{i}{\hbar}p_0(x - x_0)\right].
            \end{equation}
            \begin{enumerate}
                  \item Discuta o efeito da fase dependente de $p_0$. Qual é a sua
                        relação com o valor esperado do momento?
                  \item Calcule a evolução temporal desse estado considerando a
                        Hamiltoniana de uma partícula livre:
                        \begin{equation}
                              \hat{H} = \frac{\hat{p}^2}{2m}.
                        \end{equation}
                  \item Quais são os efeitos da propagação temporal no pacote de onda? O
                        que acontece com a a posição média e com a variância? Discuta
                        a interpretação físicas desses efeitos.
            \end{enumerate}
\end{enumerate}

\end{document}

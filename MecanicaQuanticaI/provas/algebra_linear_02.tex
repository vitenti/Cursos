\newif\ifuseseminar
\useseminartrue
\input{../../latex_common/header.tex.frag}

\title{Mecânica Quântica I -- Álgebra Linear}	

\begin{document}

\begin{enumerate}
	\item Espaços vetoriais $\mathbb{V}$:
	      \begin{enumerate}
		      \item Explique o conceito de independência linear.
		      \item Explique a distinção entre um conjunto de vetores Linearmente
		            Independentes (LI) e uma base.
		      \item Dada uma base de vetores $|e_i\rangle$ (onde $i=1\dots n$), demonstre
		            que qualquer vetor $|v\rangle$ pode sempre ser expresso como uma combinação
		            linear dos vetores da base. Qual é o nome dado aos coeficientes nesse
		            contexto?
		      \item Mostre que os coeficientes da combinação linear do item anterior são
		            únicos.
	      \end{enumerate}
	\item Espaço dual $\mathbb{V}^*$:
	      \begin{enumerate}
		      \item Dado um espaço vetorial $\mathbb{V}$, defina o seu espaço dual
		            $\mathbb{V}^*$. Os elementos deste espaço são representados como
		            $\langle w|$ e são denominados covetores.
		      \item Em um espaço vetorial $\mathbb{V}$ que possui um produto interno
		            $\left(|v\rangle, |u\rangle\right) \in
			            \mathbb{R}\;\text{ou}\;\mathbb{C}$, como podemos usar essa estrutura
		            para definir elementos de $\mathbb{V}^*$ a partir de elementos de
		            $\mathbb{V}$?
		      \item Usando o mapa definido no item anterior, temos que $|v\rangle$ pode
		            ser levado em um covetor $\langle v|$. Mostre que um vetor
		            $a|v\rangle$, onde $a\in\mathbb{C}$ é levado pelo mesmo mapa no
		            covetor $a^*\langle v|$.
		      \item O mapa do item anterior pode sempre ser definido? Explique e dê
		            exemplos para ilustrar suas explicação.
	      \end{enumerate}
	\item Operadores $\mathrm{Op}\left(\mathbb{V}\right)$:
	      \begin{enumerate}
		      \item Dado um operador linear $\hat{\Omega}$, como definimos o seu adjunto
		            $\hat{\Omega}^\dagger$?
		      \item Dado um operador auto-adjunto $\hat{\Omega}$, mostre que seus autovetores tem
		            autovalores reais. Mostre também que autovetores com autovalores diferentes
		            são sempre ortogonais.
		      \item Com base no item anterior, é possível afirmar que os autovetores
		            sempre formam uma base ortogonal?
		      \item Dada uma função suave $f:\mathbb{R}\to\mathbb{R}$, como podemos usar
		            essa função para definir uma função
		            $f:\mathrm{Op}\left(\mathbb{V}\right)\to\mathrm{Op}\left(\mathbb{V}\right)$?
	      \end{enumerate}
	\item Espaço de dimensão infinita:
	      \begin{enumerate}
		      \item Em uma base continua $|x\rangle,\;\forall\;x\in\mathbb{R}$, chamamos
		            o produto $\psi(x) = \langle x|\psi\rangle$ de função de onda.
		            Mostre que se o operador de translação $T_\epsilon$ tem a seguinte
		            ação $T_\epsilon|x\rangle = |x+\epsilon\rangle$, então $\langle
			            x|T_\epsilon|\psi\rangle = \psi(x-\epsilon)$.
		      \item Mostre que o operador de translação $T_\epsilon$ satisfaz a relação
		            $T_{\epsilon_1}T_{\epsilon_2} = T_{\epsilon_1+\epsilon_2}$.
		      \item Mostre que o adjunto do operador de translação é dado por
		            $T_\epsilon^\dagger = T_{-\epsilon}$. Podemos afirmar que o operador
		            de translação é unitário? Justifique sua resposta.
		      \item Use o resultado do item anterior mostre que o gerador $\hat{K}$,
		            definido na expressão $T_\epsilon \approx \mathbb{I} - i \epsilon
			            \hat{K}$, tem as seguintes componentes: $$\langle
			            x|\hat{K}|\psi\rangle = -i\frac{\partial}{\partial x}\psi(x).$$
	      \end{enumerate}
\end{enumerate}

\end{document}

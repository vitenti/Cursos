\newif\ifuseseminar
\useseminartrue
\input{../../latex_common/header.tex.frag}

\title{Mecânica Quântica I -- Partícula na Caixa}	

\begin{document}

Considere a Hamiltoniana de uma partícula em uma caixa unidimensional:
\begin{equation*}
  H = \frac{p^2}{2m} + V(x),
\end{equation*}
onde $V(x) = 0$ para $|x| < L/2$ e $V(x) = V_0$ para $|x| \geq L/2$.
\begin{enumerate}
  \item Considere as três regiões do espaço:
        \begin{enumerate}
          \item[I] $x \leq -L/2$,
          \item[II] $-L/2 < x < L/2$,
          \item[III] $x \geq L/2$.
        \end{enumerate}
        Encontre as soluções da equação de Schrödinger independente do tempo em cada
        região.
  \item Nas regiões I e III, quais são as soluções que são normalizáveis? O que acontece
        quando $V_0 \to \infty$?
  \item Usando o resultado do item anterior, encontre as soluções da equação de
        Schrödinger independente do tempo na região II. Mostre que as soluções são
        combinações lineares de senos e cossenos.
  \item Quais são os valores permitidos de $E$? Mostre que a energia é quantizada.
  \item Qual é o menor valor de $E$? Qual é a função de onda correspondente? Qual é
        a interpretação física?
\end{enumerate}

\end{document}

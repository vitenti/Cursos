\newif\ifuseseminar
\useseminartrue
\input{../../latex_common/header.tex.frag}

\title{Mecânica Quântica I -- Teoria de Perturbação Independente do Tempo}	

\begin{document}

Considere um sistema cuja Hamiltoniana é dada por
\begin{equation*}
      \hat{H} = \hat{H}^0 + \hat{H}^1,
\end{equation*}
onde $\hat{H}^0$ é a Hamiltoniana de um sistema cujas soluções são conhecidas,
$\hat{H}^1$ é uma perturbação fraca. A teoria de
perturbação independente do tempo visa determinar as correções nas energias e nas
funções de onda do sistema não perturbado. Em outras palavras, buscamos expressar essas
correções em termos dos autoestados de $\hat{H}^0$, que formam uma base completa
$\{\ket{n^0}\}$. Suponhamos que os autoestados sejam não degenerados, isto é:
\begin{equation*}
      \hat{H}^0\ket{n^0} = E_n^0\ket{n^0},\quad E^0_n \neq E^0_m \iff n \neq m.
\end{equation*}

Podemos supor que a diferença entre o autoestado de $\hat{H}$ e o autoestado
de $\hat{H}^0$ é pequena, isto é:
\begin{equation*}
      \ket{n} = \ket{n^0} + \ket{n^1} + \dots,
\end{equation*}
onde $\ket{n^1}$ é uma correção de primeira ordem. Analogamente, a energia
total do sistema é dada por
\begin{equation*}
      E_n = E_n^0 + E_n^1 + \dots.
\end{equation*}
\begin{enumerate}
      \item Como podemos formalizar as hipóteses acima? Em particular,
            explique como $\ket{n^1}$ pode ser compreendido em
            termos de uma série de potências.
      \item Mostre que a correção de primeira ordem na energia é dada por
            \begin{equation*}
                  E_n^1 = \bra{n^0}\hat{H}^1\ket{n^0}.
            \end{equation*}
      \item Mostre que a correção de primeira ordem na função de onda é dada por
            \begin{equation*}
                  \ket{n^1} = i\alpha\ket{n^0} + \sum_{m\neq n}\frac{\bra{m^0}\hat{H}^1\ket{n^0}}{E_n^0 - E_m^0}\ket{m^0}.
            \end{equation*}
            onde $\alpha$ é um fator real a ser determinado.
      \item Mostre como podemos remover a dependência de $\alpha$.
\end{enumerate}
\end{document}

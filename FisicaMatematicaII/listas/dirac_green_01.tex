\newif\ifuseseminar
\useseminartrue
\input{../../latex_common/header.tex.frag}

\title{Física Matemática II -- Delta de Dirac e Funções de Green}

\begin{document}
\begin{enumerate}
	\item Considere a distribuição normal
	      $p_\sigma(x)=\frac{1}{\sqrt{2\pi\sigma^2}}e^{-x^2/2\sigma^2}$, onde $\sigma>0$
	      é o desvio padrão.
	      \begin{enumerate}
		      \item Mostre que a distribuição é normalizada, i.e,
		            \begin{align}
			            \int_{-\infty}^{\infty}p_\sigma(x)\dd x = 1.
		            \end{align}
		      \item Calcule a média e o valor esperado das potências impares de $x$,
		            i.e,
		            \begin{align}
			            \left\langle x^{2n+1}\right\rangle =
			            \int_{-\infty}^{\infty}x^{2n+1}p_\sigma(x)\dd x.
		            \end{align}
		      \item Calcule a média e o valor esperado das potências pares de $x$,
		            i.e,
		            \begin{align}
			            \left\langle x^{2n}\right\rangle =
			            \int_{-\infty}^{\infty}x^{2n}p_\sigma(x)\dd x.
		            \end{align}
		      \item Usando os resultados anteriores, mostre que a integral
		            $$\int_{-\infty}^{\infty}f(x)p_\sigma(x)\dd x =
			            f(0)+\frac{\sigma^2}{2}f''(0) +
			            \frac{\sigma^4}{8}f^{(4)}(0)+\ldots.$$
		            Onde ${}'$ denota a derivada em relação a $x$.
		            Ou seja, mostre que o limite da distribuição normal quando
		            $\sigma\rightarrow 0$ é a função delta de Dirac.
	      \end{enumerate}
	\item Podemos construir a inversa da derivada de uma função $f(x)$ onde $x\in(a, b)$
	      como
	      \begin{align}
		      f(x) = \int_{a}^{x}f'(y)\dd y,\qquad f(x) = \frac{\dd}{\dd x}\int_{a}^{x}f(y)\dd y.
	      \end{align}
	      onde $f(a) = 0 = f(b)$.
	      \begin{enumerate}
		      \item Mostre que podemos reescrever as expressões acima usando a
		            função de Heaviside $\theta(x)$, i.e,
		            \begin{align}
			            \theta(x) = \begin{cases}
				                        0, & x<0,     \\
				                        1, & x\geq 0.
			                        \end{cases}
		            \end{align}
		            Ou seja, a função de Heaviside é a função de Green para a derivada.
		      \item Calcule a integral da função de Heaviside, i.e,
		            \begin{align}
			            G(x) \equiv \int_{a}^{x}\theta(y)\dd y.
		            \end{align}
		            e mostre que a integral da função de Heaviside é a função de Green
		            para a o operador derivada segunda:
		            $$\frac{\dd^2}{\dd x^2}\int_{a}^{b}G(x-y)f(y)\dd y = f(x).$$
		      \item Mostre também a relação inversa, i.e,
		            \begin{align}
			            f(x) = \int_{a}^{b}G(x-y)\frac{\dd^2}{\dd y^2}f(y)\dd y.
		            \end{align}
		      \item Escreva a função de Green em termos da distância $\sigma(x,y) = |x-y|$.
	      \end{enumerate}
	\item No caso de uma dimensão, a função de Green para o operador derivada
	      segunda é proporcional a função distância $\sigma(x,y) = |x-y|$. Em duas
	      dimensões, a função de Green também pode ser escrita em termos da
	      distância $$\sigma(\vec{x},\vec{y}) = \sqrt{(x_1-y_1)^2+(x_2-y_2)^2},$$
	      onde $\vec{x}=(x_1,x_2)$ e $\vec{y}=(y_1,y_2)$.
	      \begin{enumerate}
		      \item Mostre que o Laplaciano em duas dimensões, i.e.,
		            $$\vec\nabla^2 = \frac{\partial^2}{\partial x_1^2}+\frac{\partial^2}{\partial x_2^2},$$
		            atuando sobre $\ln\sigma(\vec{x},\vec{y})$ é zero para todo $\vec{x}\neq\vec{y}$, i.e.,
		            $$\vec\nabla^2\ln\sigma(\vec{x},\vec{y}) = 0.$$
		      \item Podemos regularizar a função $\ln\sigma(\vec{x},\vec{y})$ usando
		            o parâmetro $\varepsilon>0$, i.e,
		            $$G_\varepsilon(\vec{x},\vec{y}) = \frac{1}{2\pi}\ln\sqrt{(x_1-y_1)^2+(x_2-y_2)^2+\varepsilon^2}.$$
		            Mostre que a integral do Laplaciano da função regularizada é um, i.e,
		            $$\int_{\mathbb{R}^2}\vec\nabla^2G_\varepsilon(\vec{x},\vec{y})\dd^2\vec{x} = 1.$$
		      \item Mostre que a integral do Laplaciano da função regularizada é a função delta de Dirac,
		            i.e,
		            $$\int_{\mathbb{R}^2}\vec\nabla^2G_\varepsilon(\vec{x},\vec{y})f(\vec{y})\dd^2\vec{y} = f(\vec{x}).$$
	      \end{enumerate}
	\item Repita os passos anteriores para três dimensões, isto é:
	      \begin{enumerate}
		      \item Mostre que o Laplaciano em três dimensões, i.e.,
		            $$\vec\nabla^2 = \frac{\partial^2}{\partial x_1^2}+\frac{\partial^2}{\partial x_2^2}+\frac{\partial^2}{\partial x_3^2},$$
		            atuando sobre $1/\sigma(\vec{x},\vec{y})$ é zero para todo $\vec{x}\neq\vec{y}$, i.e.,
		            $$\vec\nabla^2\frac{1}{\sigma(\vec{x},\vec{y})} = 0.$$
		      \item Regularize a função $1/\sigma(\vec{x},\vec{y})$ usando
		            o parâmetro $\varepsilon>0$, i.e,
		            $$G_\varepsilon(\vec{x},\vec{y}) = -\frac{1}{4\pi}\frac{1}{\sqrt{(x_1-y_1)^2+(x_2-y_2)^2+(x_3-y_3)^2+\varepsilon^2}}.$$
		            Mostre que a integral do Laplaciano da função regularizada é um, i.e,
		            $$\int_{\mathbb{R}^3}\vec\nabla^2G_\varepsilon(\vec{x},\vec{y})\dd^3\vec{x} = 1.$$
		      \item Mostre que a integral do Laplaciano da função regularizada é a função delta de Dirac,
		            i.e,
		            $$\int_{\mathbb{R}^3}\vec\nabla^2G_\varepsilon(\vec{x},\vec{y})f(\vec{y})\dd^3\vec{y} = f(\vec{x}).$$
	      \end{enumerate}
	\item Faça a dedução das identidades de Green para o Laplaciano.
	      \begin{enumerate}
		      \item Primeira identidade de Green:
		            \begin{align}
			            \oint_{\partial\Omega}\phi\vec\nabla\psi\cdot\dd\vec{S} =
			            \int_{\Omega}\left(\vec\nabla\phi\cdot\vec\nabla\psi+\phi\vec\nabla^2\psi\right)\dd^3\vec{x},
		            \end{align}
		            onde $\Omega$ é um volume limitado por uma superfície $\partial\Omega$.
		      \item Segunda identidade de Green:
		            \begin{align}
			            \oint_{\partial\Omega}\left(\phi\vec\nabla\psi-\psi\vec\nabla\phi\right)\cdot\dd\vec{S} =
			            \int_{\Omega}\left(\phi\vec\nabla^2\psi-\psi\vec\nabla^2\phi\right)\dd^3\vec{x}.
		            \end{align}
	      \end{enumerate}
	\item Use as identidades de Green e a função de Green para o Laplaciano para
	      resolver a equação de Poisson em três dimensões, i.e,
	      \begin{align}
		      \vec\nabla^2\phi(\vec{x}) = -\rho(\vec{x}),
	      \end{align}
	      onde $\rho(\vec{x})$ é uma função fonte. Discuta as condições de contorno
	      para a solução da equação de Poisson.
	\item Escreva as identidades de Green para o operador derivada segunda em
	      uma dimensão. Use a função de Green para o operador derivada segunda
	      para resolver a equação de Poisson em uma dimensão, i.e,
	      \begin{align}
		      \frac{\dd^2}{\dd x^2}\phi(x) = -\rho(x),
	      \end{align}
	      onde $\rho(x)$ é uma função fonte. Discuta as condições de contorno
	      para a solução da equação de Poisson.
\end{enumerate}
\end{document}

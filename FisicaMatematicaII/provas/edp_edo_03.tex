\newif\ifuseseminar
\useseminartrue
\input{../../latex_common/header.tex.frag}

\title{Prova de Física Matemática II -- EDO e EDP}

\begin{document}
\begin{enumerate}
	\item Em quais situações é apropriado calcular soluções usando a série de Taylor e
	      em quais casos é necessário recorrer ao método de Frobenius? Dê exemplos para
	      ilustrar cada situação, incluindo casos em que o método de Frobenius não pode ser
	      aplicado.

	\item Considere a equação de Bessel:
	      \[x^2y'' + xy' + (x^2 - \nu^2)y = 0.\]
	      \begin{enumerate}
		      \item Classifique o ponto \(x_0 = 0\) e encontre a primeira solução na
		            forma de Frobenius.
		      \item Explique o tipo da segunda solução nos casos em que:
		            \begin{enumerate}
			            \item \(\nu\) é um número inteiro.
			            \item \(\nu\) é semi-inteiro.
			            \item \(2\nu \notin \mathbb{Z}\).
		            \end{enumerate}
	      \end{enumerate}
	\item Considere a equação de Legendre:
	      \[(1-x^2)y'' - 2xy' + \lambda(\lambda+1)y = 0,\] onde \(y(x)\) está definida
	      em \(x \in [-1,1]\). Resolva a equação em um dos pontos singulares regulares
	      usando o método de Frobenius e encontre duas soluções linearmente
	      independentes em torno desse ponto. Responda:
	      \begin{enumerate}
		      \item O que ocorre nos pontos singulares regulares se \(\lambda\) não for
		            um número inteiro?
		      \item Quais são as restrições sobre \(\lambda\) para garantir que as
		            soluções sejam finitas para \(x \in [-1,1]\)?
	      \end{enumerate}
\end{enumerate}

\end{document}

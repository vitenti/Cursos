\newif\ifuseseminar
\useseminartrue
\input{../../latex_common/header.tex.frag}

\title{Prova de Física Matemática II -- Átomo de Hidrogênio}

\begin{document}
\begin{enumerate}
      \item Átomo de Hidrogênio. Conceitos a Hamiltoniana representando a interação
            eletrostática entre o elétron e o próton. A equação de autovalores de
            energia e autoestados de energia é dada por:
            \begin{equation}
                  \left[ -\frac{\hbar^2}{2m} \nabla^2 - \frac{e^2}{4\pi\epsilon_0r} \right]
                  \psi_\alpha = E_\alpha \psi_\alpha,
            \end{equation}
            onde $m$ é a massa do elétron, $e$ é a carga do elétron, $r$ é a distância
            entre o elétron e o próton, $\psi$ é a função de onda do elétron e $E$ é a
            energia do elétron. Usando coordenadas esféricas, aplique o método de
            separação de variáveis para resolver a equação de autovalores de energia e
            autoestados de energia.
            \begin{enumerate}
                  \item Faça a separação de variáveis $\psi(r,\theta,\phi) =
                              R(r)Y(\theta,\phi)$. Mostre que a parte angular
                        $Y(\theta,\phi)$ pode ser escrita como harmônicos
                        esféricos fazendo a separação em $\phi$ e em $\theta$.
                        Discuta os rótulos $l$ e $m$ associados aos harmônicos
                        esféricos. Explique e discuta quais são os possíveis
                        valores.
                  \item Usando a separação de variáveis, mostre que a parte radial
                        $R(r)$ da função de onda do elétron satisfaz a equação
                        diferencial radial:
                        \begin{equation}
                              \frac{1}{r^2} \frac{\dd}{\dd r} \left( r^2 \frac{\dd R}{\dd r} \right) +
                              \left[ \frac{2m}{\hbar^2} \left( E - \frac{e^2}{4\pi\epsilon_0r} \right) -
                                    \frac{l(l+1)}{r^2} \right] R = 0.
                        \end{equation}
                  \item Identifique as escalas físicas relevantes para a equação acima e
                        reescreva a equação em termos adimensionais.
                  \item Estude o limite assintótico em $r$ da equação radial. Discuta as
                        soluções para $r \to 0$ e $r \to \infty$. Discuta a relação
                        entre o limite assintótico e a condição de normalização da
                        função de onda. Discuta também a relação do limite com o sinal
                        da energia.
                  \item Especialize a equação radial para o caso convergente. Encontre
                        as soluções que satisfazem a condição de normalização. Aplique o
                        método de Frobenius para encontrar a solução geral da equação
                        radial. Mostre e discuta os valores da energia permitidos.
            \end{enumerate}
\end{enumerate}

\end{document}

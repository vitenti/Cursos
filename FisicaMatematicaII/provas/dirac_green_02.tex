\newif\ifuseseminar
\useseminartrue
\input{../../latex_common/header.tex.frag}

\title{Física Matemática II -- Delta de Dirac e Funções de Green}

\begin{document}
\begin{enumerate}
	\item Podemos construir a inversa da derivada de uma função $f(x)$ onde $x\in(a, b)$
	      como
	      \begin{align}
		      f(x) = \int_{a}^{x}f'(y)\dd y,\qquad f(x) = \frac{\dd}{\dd x}\int_{a}^{x}f(y)\dd y.
	      \end{align}
	      onde $f(a) = 0 = f(b)$.
	      \begin{enumerate}
		      \item Mostre que podemos reescrever as expressões acima usando a
		            função de Heaviside $\theta(x)$, i.e,
		            \begin{align}
			            \theta(x) = \begin{cases}
				                        0, & x<0,     \\
				                        1, & x\geq 0.
			                        \end{cases}
		            \end{align}
		            Ou seja, a função de Heaviside é a função de Green para a derivada.
		      \item Calcule a integral da função de Heaviside, i.e,
		            \begin{align}
			            G(x) \equiv \int_{a}^{x}\theta(y)\dd y.
		            \end{align}
		            e mostre que a integral da função de Heaviside é a função de Green
		            para a o operador derivada segunda:
		            $$\frac{\dd^2}{\dd x^2}\int_{a}^{b}G(x-y)f(y)\dd y = f(x).$$
		      \item Mostre também a relação inversa, i.e,
		            \begin{align}
			            f(x) = \int_{a}^{b}G(x-y)\frac{\dd^2}{\dd y^2}f(y)\dd y.
		            \end{align}
		      \item Escreva a função de Green em termos da distância $\sigma(x,y) = |x-y|$.
	      \end{enumerate}
	\item Funções de Green Unidimensionais.
	      \begin{enumerate}
		      \item Deduza as identidades de Green unidimensionais para o operador
		            diferencial $\hat{L} = -\frac{d^2}{dx^2}$ no intervalo $(a, b)$ para
		            uma classe de funções que satisfaça $f(a) = 0 = f(b)$. Para isso, use
		            a relação
		            $$\frac{\dd}{\dd x}\left(\phi \frac{\dd\psi}{\dd x}\right) =
			            \frac{\dd\phi}{\dd x}\frac{\dd\psi}{\dd x} + \phi\frac{\dd^2\psi}{\dd x^2},
		            $$
		            para encontrar os equivalentes unidimensionais de
		            \begin{align}
			            \oint_{\partial\Omega}\phi\vec\nabla\psi\cdot\dd\vec{S}                                 & =
			            \int_{\Omega}\left(\vec\nabla\phi\cdot\vec\nabla\psi+\phi\vec\nabla^2\psi\right)\dd^3\vec{x}, \\
			            \oint_{\partial\Omega}\left(\phi\vec\nabla\psi-\psi\vec\nabla\phi\right)\cdot\dd\vec{S} & =
			            \int_{\Omega}\left(\phi\vec\nabla^2\psi-\psi\vec\nabla^2\phi\right)\dd^3\vec{x}.
		            \end{align}
		      \item Usando a questão anterior, mostre que a função de Green para o
		            operador em questão é
		            $$G(x-y) = \frac{|x-y|}{2}+\alpha (x-y) + \beta.$$
		      \item Para achar $\alpha$ e $\beta$, use a condição de contorno
		            $G(x-y) = 0$ para $x = a$, $x = b$, $y > a$ e $y < b$. Encontre as funções $\alpha(y)$
		            e $\beta(y)$.
		      \item Usando a função de Green, mostre que a solução da equação
		            diferencial $-\frac{\dd^2}{\dd x^2}f(x) = 1$ é
		            $f(x) = -(x-a)(x-b).$
	      \end{enumerate}
\end{enumerate}
\end{document}

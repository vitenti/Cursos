\newif\ifuseseminar
\useseminartrue
\input{../../latex_common/header.tex.frag}

\title{Prova de Física Matemática II -- Geometria Diferencial}

\begin{document}

\begin{enumerate}
	\item Em uma variedade $M$ podemos escrever um campo vetorial como uma derivação,
	      i.e.,
	      \begin{equation}
		      \bar{v} = v^\mu \frac{\partial}{\partial x^\mu},
	      \end{equation}
	      onde $x^\mu$ são as coordenadas em uma carta $(\psi, U)$.
	      \begin{enumerate}
		      \item O que acontece com as componentes do campo vetorial se mudarmos para
		            uma segunda carta $(\phi,U)$ com coordenadas $y^\mu$ definida no mesmo
		            aberto $U$.
		      \item Como se transformam as um-formas $\tilde{w}=w_\mu\dd x^\mu$ quando
		            fazemos a mesma mudança de carta do item anterior?
	      \end{enumerate}
	\item Os seguintes campos vetoriais formam uma base nos espaços tangentes do espaço
	      euclidiano bidimensional $\mathbb{E}^2$:
	      \begin{equation}
		      \bar{r} = \cos\theta \frac{\partial}{\partial x}+\sin\theta
		      \frac{\partial}{\partial y}, \qquad \bar{\theta} = -\sin\theta
		      \frac{\partial}{\partial x}+\cos\theta \frac{\partial}{\partial y}.
	      \end{equation}
	      Onde usamos a mudança de cartas $x=r\cos\theta$ e $y=r\sin\theta$..
	      \begin{enumerate}
		      \item Mostre que de fato eles são linearmente independentes.
		      \item A base é bem definida em todos os pontos de $\mathbb{E}^2$? Justifique.
		      \item Mostre se essa base é coordenada.
	      \end{enumerate}
	\item Resolva a equação de Bessel abaixo usando o método de Frobenius, encontre as
	      duas soluções linearmente independentes, considere que $\alpha \in \mathbb{Z}$
	      (conjunto dos inteiros). faça também os itens abaixo.
	      \begin{equation}\label{eq:bessel}
		      x^2\frac{\dd^2 y(x)}{\dd x^2} + x\frac{\dd y(x)}{\dd x}+(x^2-\alpha^2)y(x) = 0.
	      \end{equation}
	      \begin{enumerate}
		      \item Classifique os pontos do intervalo de $x\in\mathbb{R}$ (inclua
		            também o ponto no infinito) como ordinário, singular regular ou
		            singular irregular.
		      \item O que mudaria no método de solução se $\alpha$ fosse um número real
		            diferente de inteiros e semi-inteiros.
	      \end{enumerate}
	\item Usando a teoria de Sturm-Louville faça as seguintes atividades:
	      \begin{enumerate}
		      \item Escreva o operador diferencial associado a equação de
		            Bessel~\eqref{eq:bessel}.
		      \item Usando o produto interno:
		            \begin{equation}
			            \langle f|g\rangle = \int_0^L f^*(x)g(x)w(x)\dd x,
		            \end{equation}
		            determine qual deve ser a função peso $w(x)$ para que o operador do
		            item anterior possa ser auto-adjunto.
		      \item Uma vez encontrada a função peso $w(x)$ apropriada, quais condições
		            sobre as soluções da Eq.~\eqref{eq:bessel} para que o operador seja
		            auto-adjunto.
	      \end{enumerate}
\end{enumerate}

\end{document}
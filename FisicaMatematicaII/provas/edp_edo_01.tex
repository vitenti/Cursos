\newif\ifuseseminar
\useseminartrue
\input{../../latex_common/header.tex.frag}

\title{Prova de Física Matemática II -- EDO e EDP}

\begin{document}

\begin{enumerate}
	\item  Escreva o problema de auto-valor para a Hamiltoniana do átomo de Hidrogênio e
	      desenvolva as seguintes questões:
	      \begin{enumerate}
		      \item Usando coordenadas esféricas, resolva a parte angular do problema e
		            descreva os auto-valores e auto-funções necessárias.
		      \item Escreva a equação diferencial ordinária para a parte radial e
		            identifique as escalas do problema.
		      \item Estude o comportamento assimptótico das soluções radiais.
		      \item Use o resultado de (c) para reescrever a equação da parte radial de
		            forma	compatível com soluções que vão a zero no infinito.
	      \end{enumerate}
	\item Use o método da fórmula de Rodrigues para encontrar as soluções polinomiais da
	      equação de Laguerre.  Faça também os itens abaixo:
	      \begin{enumerate}
		      \item Utilize o resultado para escrever uma solução da parte radial do
		            problema de auto-valor do átomo de Hidrogênio.
		      \item Com a fórmula de Rodrigues mostre que os polinômios de Laguerre são
		            ortogonais no intervalo $(0,\infty)$.
	      \end{enumerate}
	\item Repita os passos feitos em aula e faça a deduções da função de Green retardada
	      do operador d'Alambertiano.
\end{enumerate}

\end{document}

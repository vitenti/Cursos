\newif\ifuseseminar
\useseminartrue
\input{../../latex_common/header.tex.frag}

\title{Prova de Física Matemática II -- Geometria Diferencial}

\begin{document}

\begin{enumerate}
	\item Dado um círculo unitário definido por $S^1=\left\{(x,y)\in \mathbb{R}^2;
		      x^2+y^2=1\right\}$, mostre explicitamente que $S^1$ é uma variedade
	      diferenciável. Para isso faça os itens abaixo:
	      \begin{enumerate}
		      \item Encontre um conjunto mínimo de cartas $(U_i,\,\phi_i)$ para $0<i<m$
		            (onde $m$ é o número de cartas) e mostre que ele cobre toda a variedade.
		      \item Mostre que as funções $\phi_1\circ\phi_2^{-1}$ e
		            $\phi_2\circ\phi_1^{-1}$ são suaves em $U_1 \cap U_2$.
	      \end{enumerate}

	\item Seja o mapa $h:\mathbb{R}^2\rightarrow\mathbb{R}^3$ dado por
	      \begin{align}
		      h(x,y) = \left(\begin{matrix}
			                     u(x,y) \\
			                     v(x,y) \\
			                     w(x,y) \\
		                     \end{matrix}\right)
		      =\left(\begin{matrix}
			             x       \\
			             y       \\
			             x^2+y^2 \\
		             \end{matrix}\right)
	      \end{align}
	      E um campo vetorial $$V=x\frac{\partial}{\partial x}+y\frac{\partial}{\partial
			      y},$$ calcule o campo em $\mathbb{R}^3$ definido pelo
	      \emph{push-forward} $h_* V\in T_{h(x,y)}\mathbb{R}^3$ (lembre-se que
	      $h_* V$ é uma derivação em $C^\infty\left(\mathbb{R}^3\right)$).
	      \textbf{Item bonus} (ponto extra na prova), calcule as curvas
	      integrais de $V$, ou seja, as curvas cuja a tangente coincide com $V$
	      ao longo de suas trajetórias.

	\item Em quais situações podemos calcular soluções usando a série de Taylor e em
	      quais é necessário usar o método de Frobenius? Em quais casos o método de
	      Frobenius \textbf{não} pode ser aplicado? Dê um exemplo para cada uma dessas
	      situações.

	\item  Dada a equação de Legendre
	      $$(1-x^2)y^{\prime\prime}-2xy^{\prime}+\lambda(\lambda+1)y=0,$$ onde $y(x)$
	      está definida em $x\in[-1,1]$, resolva a equação em um dos pontos singulares
	      regulares usando o método de Frobenius e encontre duas soluções linearmente
	      independentes em torno desse ponto. Responda as perguntas:
	      \begin{enumerate}
		      \item O que acontece nos pontos singulares regulares se $\lambda$ não for
		            um número inteiro?
		      \item Quais restrições temos sobre $\lambda$?
	      \end{enumerate}
\end{enumerate}

\end{document}

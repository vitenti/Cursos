\newif\ifuseseminar
\useseminartrue
\input{../../latex_common/header.tex.frag}

\title{Prova de Física Matemática II -- EDO E EDP}

\begin{document}
\begin{enumerate}

	\item Considere a equação do oscilador harmônico:
	      \[y'' + \omega^2 y = 0,\]
	      onde \(\omega\) é uma constante real positiva.
	      Mostre usando séries de potência que a solução geral da equação acima é dada por:
	      \[y(x) = A \cos(\omega x) + B \sin(\omega x),\]
	      onde \(A\) e \(B\) são constantes reais.

	\item Para aplicar o método de Frobenius é necessário fazer primeiro a
	      classificação dos pontos singulares da equação diferencial. Supondo que o
	      dominio das equações diferenciais seja $\mathbb{R}$, classifique os pontos
	      singulares, quando presentes, das equações diferenciais abaixo:
	      \begin{enumerate}
		      \item $x^2y'' + xy' + y = 0$.
		      \item $x^2y'' + x^{3/2}y = 0$.
		      \item $y'' + x y' + \frac{y}{e^x - 1 - x} = 0$.
	      \end{enumerate}
	\item Considere a equação de Bessel:
	      \[x^2y'' + xy' + (x^2 - \nu^2)y = 0.\]
	      \begin{enumerate}
		      \item Classifique o ponto \(x_0 = 0\) e encontre a primeira solução na forma de Frobenius.
		      \item Explique como encontrar a segunda solução linearmente independente, não
		            é necessário resolver a equação.
		            \begin{enumerate}
			            \item \(\nu\) é um número inteiro.
			            \item \(\nu\) é semi-inteiro.
			            \item \(2\nu \notin \mathbb{Z}\).
		            \end{enumerate}
	      \end{enumerate}

	\item Considere a equação de Legendre:
	      \[(1-x^2)y'' - 2xy' + \lambda(\lambda+1)y = 0,\]
	      onde \(y(x)\) está definida em \(x \in [-1,1]\). Para resolver a equação em $x_0=1$
	      usando o método de Frobenius siga os passos abaixo:
	      \begin{enumerate}
		      \item Classifique o ponto $x_0 = 1$.
		      \item Usando a série de Frobenius, encontre o polinômio indicial e suas raízes.
		      \item Encontre a relação de recorrência para os coeficientes da série de Frobenius.
		      \item Resolva a relação de recorrência para encontrar a primeira solução.
		      \item Descreva como encontrar a segunda solução linearmente independente.
	      \end{enumerate}

\end{enumerate}

\end{document}

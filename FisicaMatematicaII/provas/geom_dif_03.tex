\newif\ifuseseminar
\useseminartrue
\input{../../latex_common/header.tex.frag}

\title{Prova de Física Matemática II -- Geometria Diferencial}

\begin{document}
\begin{enumerate}
	\item Dado um círculo unitário definido por $S^1=\left\{(x,y)\in \mathbb{R}^2;
		      x^2+y^2=1\right\}$, mostre explicitamente que $S^1$ é uma variedade
	      diferenciável. Para isso faça os itens abaixo:
	      \begin{enumerate}
		      \item Encontre um conjunto minimo de cartas $(U_i,\,\phi_i)$ para $0<i<m$
		            (onde $m$ é o número de cartas) e mostre que ele cobre toda a variedade.
		      \item Mostre que as funções $\phi_1\circ\phi_2^{-1}$ e
		            $\phi_2\circ\phi_1^{-1}$ são suaves em $U_1 \cap U_2$.
	      \end{enumerate}

	\item Seja o mapa $h:\mathbb{R}^2\rightarrow\mathbb{R}^3$ dado por
	      \begin{align}
		      h(x,y) = \left(\begin{matrix}
			                     u(x,y) \\
			                     v(x,y) \\
			                     w(x,y) \\
		                     \end{matrix}\right)
		      =\left(\begin{matrix}
			             x       \\
			             y       \\
			             x^2+y^2 \\
		             \end{matrix}\right)
	      \end{align}
	      E um campo vetorial $$V=x\frac{\partial}{\partial x}+y\frac{\partial}{\partial
			      y},$$ calcule o campo em $\mathbb{R}^3$ definido pelo \emph{push-forward} $h_*
		      V\in T_{h(x,y)}\mathbb{R}^3$ (lembre-se que $h_* V$ é uma derivação em
	      $C^\infty\left(\mathbb{R}^3\right)$).
	\item Calcule as curvas integrais do campo vetorial $V$ dado na questão anterior, ou
	      seja, as curvas cuja tangente coincide com $V$ ao longo de suas trajetórias,
	      $$\sigma_*\frac{\mathrm{d}}{\mathrm{d}t} = V_\sigma.$$.
	\item Prove que a aplicação de duas derivações na mesma função não forma uma
	      derivação. Em seguida, prove que o comutador de duas derivações, ou seja,
	      $$[V,U](f)\equiv V(U(f))-U(V(f))$$ é uma derivação ($V,U$ são derivações e $f$ uma
	      função da variedade).
\end{enumerate}

\end{document}

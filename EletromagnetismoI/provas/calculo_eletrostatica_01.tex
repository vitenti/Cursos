\newif\ifuseseminar
\useseminartrue
\input{../../latex_common/header.tex.frag}

\title{Eletromagnetismo I -- Cálculo e Eletrostática}	

\begin{document}
\begin{enumerate}
	\item Considere as coordenadas parabólicas cilíndricas:
	      \begin{equation}
		      x = \sigma\tau\cos\phi, \qquad y = \sigma\tau\sin\phi,
		      \qquad z = \frac{1}{2}\left(\tau^2-\sigma^2\right).
	      \end{equation}
	      \begin{enumerate}
		      \item Encontre o elemento de volume de integração tridimensional $\dd^3x$
		            nessas coordenadas.
		      \item Encontre o laplaciano $\vec\nabla^2$ nessas coordenadas.
		      \item Aplique a separação de variáveis para a equação de Laplace
		            $\vec\nabla^2V=0$ nessas coordenadas e encontre as três equações
		            diferenciais ordinárias associadas a cada variável. Determine os
		            auto-valores da equação para a variável $\phi$.
	      \end{enumerate}
	\item Eletrostática:
	      \begin{enumerate}
		      \item Quais características do campo eletrostático $\vec E$ possibilitaram
		            a introdução de um potencial eletrostático $V$?
		      \item Dada uma distribuição de carga superficial $\sigma$, mostre o efeito
		            de tal distribuição no campo elétrico calculado sobre a superfície. O que
		            acontece com o potencial eletrostático?
		      \item Em um problema dado em um volume $\mathcal{V}$ e distribuição de
		            carga $\rho$, sob quais condições o potencial $V$ é unicamente definido?
		            Em que situações o potencial é definido a menos de uma constante?
		      \item Dado um condutor com uma cavidade interna, como as cargas externas
		            ao condutor influenciam o campo elétrico dentro das cavidades? O que
		            acontece no caso contrário, ou seja, como cargas dentro das cavidades
		            influenciam o campo elétrico fora do condutor? Justifique suas respostas.
	      \end{enumerate}
\end{enumerate}
\end{document}
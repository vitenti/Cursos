\newif\ifuseseminar
\useseminartrue
\input{../../latex_common/header.tex.frag}

\title{Eletromagnetismo I -- Cálculo e Eletrostática}	

\begin{document}

\begin{enumerate}
	\item Considere uma esfera dielétrica uniformemente carregada de raio $a$.
	      \begin{enumerate}
		      \item Encontre o campo elétrico dentro e fora da esfera usando a Lei de Gauss.
		      \item Encontre o potencial e o campo elétrico dentro e fora da esfera
		            usando as equações de Poisson e Laplace.
	      \end{enumerate}
	\item Dado uma partícula de carga $q_1$ na posição $\vec{x}_1 = (a, 0, 0)$ e uma
	      placa condutora infinita e aterrada no plano $y \times z$, faça:
	      \begin{enumerate}
		      \item Calcule o potencial elétrico no volume definido por $\mathcal{V} =
			            \{\vec{x}\in\mathbb{R}^3|x^1>0\}$ usando o método das imagens.
		      \item Uma segunda carga $q_2$ é colocada na posição $\vec{x}_2 = (a, b,
			            0)$, calcule o potencial elétrico em $\mathcal{V}$.
		      \item Calcule a carga superficial induzida na placa nas duas configurações
		            acima. \textbf{Bonus}, é possível generalizar a solução do potencial
		            para uma distribuição $\rho$ definida em $\mathcal{V}$? Como seria o
		            resultado?
	      \end{enumerate}
	\item Considere dois cilindros condutores concêntricos,
	      \begin{enumerate}
		      \item Deduza o operador Laplaciano em coordenadas cilíndricas.
		      \item Considerando que não há cargas entre os cilindros, resolva a equação
		            de Laplace na região entre eles usando o método da separação de variáveis.
	      \end{enumerate}
	\item Em um campo elétrico $E = E_0\vec{e}_3$ é colocada uma esfera condutora de
	      raio $R$ e carga $q_t$ de forma que ela distorce o campo em sua proximidade.
	      Calcule o potencial eletrostático $V$ no exterior da esfera. Considere $V =
		      V_0(r) + V_1(r)\cos\theta$, utilize as condições de contorno e a equação de
	      Laplace.
\end{enumerate}
\end{document}

\newif\ifuseseminar
\useseminartrue
\input{../../latex_common/header.tex.frag}

\title{Eletromagnetismo I -- Meios Materiais}	

\begin{document}

\begin{enumerate}
	\item  Descreva as justificativas físicas e matemáticas necessárias para a
	      utilização de um campo de polarização $\vec{P}$ na descrição de materiais
	      dielétricos. Na sua resposta explique também qual é o significado do campo elétrico
	      $\vec{E}$ nesse contexto.
	\item Mostre que o dipolo magnético de uma espira é proporcional  a ``área
	      vetorial'' definida por seu contorno. Defina ``área vetorial'' e encontre sua
	      relação com a integral de linha sobre sua borda .
	\item Explique porque o rotacional da magnetização $\vec{M}$ pode ser interpretado
	      como uma corrente ``ligada'' no material.
	\item Faça a demonstração de que uma espira infinitesimal na presença de uma campo
	      magnético não uniforme $\vec{B}$, experimenta uma força da forma
	      \begin{equation}
		      \vec{F} = \vec\nabla\left(\vec{m}\cdot\vec{B}\right).
	      \end{equation}
\end{enumerate}

\end{document}